\chapter{Introduzione}


Gli stimoli meccanici rivestono nell'ambito dei sistemi biologici un
ruolo importante nel determinare il corretto funzionamento di cellule,
tessuti e organismi complessi.

Mentre tradizionalmente la Biologia si è occupata di
studiare come processi cellulari e inter-cellulari fossero regolati
dallo scambio di molecole biologiche, il ruolo degli stimoli
meccanici è stato a lungo ritenuto marginale nella descrizione di
questi processi.

Lo sviluppo di tecniche sempre più avanzate e precise per la
visualizzazione e la manipolazione di molecole all'interno di campioni
biologici ha iniziato a mutare questa concezione: oggi possiamo
indagare nel dettaglio il funzionamento dei motori
molecolari all'interno delle nostre cellule o
misurare come variazioni nella tensione applicata ad un polimero
possano indurre una riorganizzazione strutturale nello stesso e 
cambiarne le proprietà biochimiche.

Per molti processi biologici il ruolo della forza è fondamentale,
ad esempio nei complessi proteici che legano tra di loro le cellule
in un tessuto. Questi si comportano come complesse macchine in grado
di elaborare stimoli di tipo biochimico e meccanico, comunicando e 
interferendo con il resto delle funzioni cellulari.

Le pinzette ottiche permettono di sondare il comportamento di
complessi proteici sottoposti a stimoli meccanici variabili, osservando
ad esempio come questi posssano modulare l'interazione tra due proteine
diverse. La teoria alla base del loro funzionamento è introdotta nella sezione \ref{sec:ot}.

Quando sono combinate con tecniche ultraveloci per il posizionamento
delle trappole e il rilevamento degli spostamenti degli oggetti intrappolati le pinzette ottiche consentono la realizzazione di esperimenti di \emph{spettroscopia force-clamp}, approfonditi nella sezione \ref{sec:force_clamp}.

Parallelamente la microscopia ottica ha permesso di descrivere i processi biologici con una precisione sempre maggiore, rendendo possibile la rilevazione e il tracciamento di singole molecole. In particolare
nell'ambito della microscopia di fluorescenza sono state sviluppate
tecniche per ricostruire immagini superando il \emph{limite di diffrazione}, per indurre la produzione di proteine fluorescenti grazie
all'ingegneria genetica, per rendere rilevabile il segnale di singoli fluorofori immobilizzati sopprimendo il rumore di quelli liberi in
soluzione. La teoria alla base di alcune di queste techniche è introdotta nella sezione \ref{sec:imaging}.

Lo scopo di questa tesi è combinare un sistema di \emph{spettroscopia force-clamp} con un sistema di \emph{imaging di singola molecola} per l'esecuzione di misure in vitro simultanee e sincronizzate.

In questo modo sarà possibile studiare, in un ambiente
controllato (proteine in soluzione e immobilizzate su microsfere),
il comportamento di proteine \emph{meccano-sensibili}, unendo alle
informazioni meccaniche quelle sulla dinamica di interazione
con altri fattori opportunamente marcarti presenti in soluzione.



% Introduction on the importance of mechanotransduction



 %%%%%%%%% %%%%%%%%% %%%%%%%%% %%%%%%%%% %%%%%%%%% %%%%%%%%% %%%%%%%%%



% between 

\section{Pinzette ottiche}
\label{sec:ot}


Le pinzette ottiche (o \textit{optical tweezers}, OT) sono strumenti che sfruttano la \emph{forza di radiazione} esercitata da un fascio laser gaussiano altamente focalizzato su materiali dielettrici, in modo da intrappolare e manipolare oggetti microscopici con una precisione sub-nanometrica.

Questa tecnologia sfrutta il gradiente d'intensità di un fascio gaussiano focalizzato in prossimità del suo \textit{waist} e l'interazione tra il dipolo elettrico indotto nel materiale e il fascio.

Arthur Ashkin fu, nel 1986, il primo a realizzare sperimentalmente delle pinzette ottiche, riuscendo a intrappolare microsfere sintetiche e batteri\cite{Ashkin:86}. Per questo risultato gli fu conferito il premio Nobel nel 2018, \emph{``per le pinzette ottiche e le loro applicazioni ai sistemi biologici''}.

Per descrivere quantitativamente il funzionamento delle pinzette ottiche possiamo considerare l'interazione radiazione-materia nel limite di oggetti molto più piccoli della lunghezza d'onda della radiazione.

In questo limite possiamo considerare il materiale interagente con la radiazione come un dipolo elettrico puntiforme, associato ad una polarizzabilità $\alpha$. Il vettore di polarizzazione nel dipolo puntiforme sarà quindi $\vec{p} = \alpha \vec{E}$.

La forze esercitata su un dipolo elettrico puntiforme può essere ricavata a partire dalle \emph{legge di Lorentz}, ottenendo:

$$ \vec{F} =
  \left( \vec{p} \cdot \vec{\nabla} \right) \vec{E}
  + \frac{d\vec{p}}{dt} \times \vec{B}
$$

Ovvero, una volta sostituito il vettore di polarizzazione:

$$ \vec{F} = \alpha
\left[
    \left( \vec{E} \cdot \vec{\nabla} \right) \vec{E}
    + \frac{d\vec{E}}{dt} \times \vec{B}
\right]
$$

E infine, tenendo conto delle \emph{equazioni di Maxwell} e dell'algebra dei vettori:

\begin{equation}
\label{dipole_force}
\vec{F}
= \alpha 
\left[
    \frac{1}{2}\nabla E^2
    + \frac{d}{dt}\left(\vec{E} \times \vec{B}\right)
\right]
\end{equation}

Questa ultima forma (equazione \ref{dipole_force}) ci permette di mettere in evidenza il termine $\frac{d}{dt}(\vec{E} \times \vec{B})$, ovvero la derivata temporale di una quantità oscillante alla stessa frequenza ottica del fascio laser (\SI{> 1e14}{\Hz}).

Confrontando questo valore con la frequenza con cui riusciamo a campionare sperimentalmente il valore della forza risulta accurato considerare questa quantità costante, e quindi trascurare il secondo termine.


\section{Spettroscopia force-clamp}

