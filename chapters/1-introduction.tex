 %%%%%%%%% %%%%%%%%% %%%%%%%%% %%%%%%%%% %%%%%%%%% %%%%%%%%% %%%%%%%%%

\chapter{Introduzione}


Gli stimoli meccanici rivestono nell'ambito dei sistemi biologici un
ruolo importante nel determinare il corretto funzionamento di cellule,
tessuti e organismi complessi.

Mentre tradizionalmente la biologia si è occupata di
studiare come processi cellulari e inter-cellulari fossero regolati
dallo scambio di molecole biologiche, il ruolo degli stimoli
meccanici è stato a lungo ritenuto marginale nella descrizione di
questi processi.

Lo sviluppo di tecniche sempre più avanzate e precise per la
visualizzazione e la manipolazione di molecole all'interno di campioni
biologici ha iniziato a mutare questa concezione: oggi possiamo
indagare nel dettaglio il funzionamento dei motori molecolari
all'interno delle nostre cellule o misurare come variazioni nella
tensione applicata a un polimero possano indurre una riorganizzazione
strutturale nello stesso e cambiarne le proprietà biochimiche.

Per molti processi biologici il ruolo della forza è fondamentale,
ad esempio nei complessi proteici che legano tra di loro le cellule
in un tessuto, le \emph{giunzioni cellulari}.
Queste si comportano come complesse macchine in grado di elaborare
stimoli di tipo biochimico e meccanico, comunicando e interferendo
con le funzioni del resto della cellula.
Esistono diversi tipi di giunzioni cellulari, responsabili di
specifiche funzioni e caratterizzate dalla reciproca interazione di
diversi tipi di proteine. La dinamica della loro interazione viene
modificata e modulata dalle sollecitazioni meccaniche esterne,
permettendo alle giunzioni in \emph{trasduttori} di segnali meccanici.

Le pinzette ottiche permettono di sondare il comportamento di
complessi proteici sottoposti a stimoli meccanici variabili,
osservando
ad esempio come questi posssano modulare l'interazione tra due
proteine diverse. La teoria alla base del loro funzionamento è
introdotta nella sezione \ref{sec:ot}.

Quando sono combinate con tecniche ultraveloci per il posizionamento
delle trappole e il rilevamento degli spostamenti degli oggetti
intrappolati le pinzette ottiche consentono la realizzazione di
esperimenti di \emph{spettroscopia force-clamp}, approfonditi nella
sezione \ref{sec:force_clamp}.

Parallelamente la microscopia ottica ha permesso di descrivere i
processi biologici con una precisione sempre maggiore, rendendo
possibile la rilevazione e il tracciamento di singole molecole.
In particolare nell'ambito della microscopia di fluorescenza sono
state sviluppate tecniche per ricostruire immagini superando il
\emph{limite di diffrazione}, per indurre la produzione di proteine
fluorescenti grazie all'ingegneria genetica, per rendere rilevabile
il segnale di singoli fluorofori immobilizzati sopprimendo il rumore
di quelli liberi in soluzione.
La teoria alla base di alcune di queste techniche è introdotta
nella sezione \ref{sec:imaging}.

Lo scopo di questa tesi è combinare un sistema di \emph{spettroscopia force-clamp} con un sistema di \emph{imaging di singola molecola} per l'esecuzione di misure in vitro simultanee e sincronizzate.

In questo modo sarà possibile studiare, in un ambiente
controllato (proteine in soluzione e immobilizzate su microsfere),
il comportamento di proteine \emph{meccano-sensibili}, unendo alle
informazioni meccaniche quelle sulla dinamica di interazione
con altri fattori opportunamente marcarti presenti in soluzione.



% Introduction on the importance of mechanotransduction



 %%%%%%%%% %%%%%%%%% %%%%%%%%% %%%%%%%%% %%%%%%%%% %%%%%%%%% %%%%%%%%%



% between 

\section{Pinzette ottiche}
\label{sec:ot}


Le pinzette ottiche (o \textit{optical tweezers}, OT) sono strumenti che sfruttano la \emph{forza di radiazione} esercitata da un fascio laser gaussiano altamente focalizzato su materiali dielettrici, in modo da intrappolare e manipolare oggetti microscopici con una precisione sub-nanometrica.

Questa tecnologia sfrutta il gradiente d'intensità di un fascio
gaussiano focalizzato interagente con particelle dielettriche immerse
in un fluido. L'interazione delle particelle con la radiazione fa si
che queste risentano di una forza di richiamo verso una posizione
di equilibrio in prossimità del fuoco del fascio.

Arthur Ashkin fu, nel 1986, il primo a realizzare sperimentalmente delle pinzette ottiche, riuscendo a intrappolare microsfere sintetiche e batteri\cite{Ashkin:86}. Per questo risultato gli fu conferito il premio Nobel nel 2018, \emph{``per le pinzette ottiche e le loro applicazioni ai sistemi biologici''}.

Per descrivere quantitativamente il funzionamento delle pinzette
ottiche consideriamo in generale l'effetto dell'interazione tra
una microsfera dielettrica, immersa in una soluzione liquida, e
la radiazione elettromagnetica prodotta da un fascio laser gaussiano
focalizzato.

In generale la forza a cui è soggetta la microsfera interagente
col campo elettromagnetico può essere scomposta in due contributi:

\begin{itemize}
    \item La \textbf{forza di \textit{scattering}} o pressione di radiazione, sempre orientata nella direzione di propagazione
    della radiazione e proporzionale alla sua intesità.
    \item La \textbf{forza di dipolo} o gradiente, proporzionale
    al gradiente d'intensità della radiazione elettromagnetico.
\end{itemize}

L'origine di questi due contributi e la dipenza dalle caratteristiche
della microsfera e del liquido utilizzati possono essere derivate
analiticamente dalle equazioni di Maxwell nei limiti del regime
di Rayleigh, ovvero quando le dimensioni della sfera sono molto
inferiori alla lunghezza d'onda della radiazione utilizzata.

In questo limite possiamo considerare il materiale interagente con la
radiazione come un dipolo elettrico puntiforme, associato ad una
polarizzabilità $\alpha$. Il vettore di polarizzazione nel dipolo puntiforme sarà quindi $\vec{p} = \alpha \vec{E}$.

La pressione di radiazione sarà quindi proporzionale all'impulso
dei fotoni retrodiffusi per \textit{scattering} Rayleigh.
Nel caso di una microsfera di raggio $a$, indice di rifrazione $n$,
immersa in un fluido con indice di rifrazione $m$, la forza di
\textit{scattering} può essere espressa\cite{HARADA1996529} come:

\begin{equation}
\vec{F}_r = \hat{k} \frac{8 \pi n k^4 a^6}{3c}
\left(
\frac{(n/m)^2 - 1}{(n/m)^2 + 2}
\right)^2
\end{equation}

L'espressione della forza gradiente può essere ottenuta dall'interazione
lorentziana tra la radiazione e il dipolo puntiforme:
L
$$ \vec{F}_g =
  \left( \vec{p} \cdot \vec{\nabla} \right) \vec{E}
  + \frac{d\vec{p}}{dt} \times \vec{B}
$$

Ovvero, una volta sostituito il vettore di polarizzazione:

$$ \vec{F}_g = \alpha
\left[
    \left( \vec{E} \cdot \vec{\nabla} \right) \vec{E}
    + \frac{d\vec{E}}{dt} \times \vec{B}
\right]
$$

E infine, tenendo conto delle \emph{equazioni di Maxwell} e dell'algebra dei vettori:

\begin{equation}
\label{dipole_force}
\vec{F_g}
= \alpha 
\left[
    \frac{1}{2}\nabla E^2
    + \frac{d}{dt}\left(\vec{E} \times \vec{B}\right)
\right]
\end{equation}

Questa ultima forma (equazione \ref{dipole_force}) ci permette di mettere in evidenza il termine $\frac{d}{dt}(\vec{E} \times \vec{B})$, ovvero la derivata temporale di una quantità oscillante molto rapidamente (\SI{> 1e14}{\Hz}), che
può tranquillamente essere considerata costante se confrontata con in tempi
tipici dell'evoluzione meccanica del sistema. Il secondo termine può quindi
essere trascurato e, sostituendo ad $\alpha$ l'espressione per la polarizzabilità
della microsfera otteniamo:

\begin{equation}
    \vec{F}_g = 
    \frac{2\pi n a^3}{c}
    \left(
    \frac{(n/m)^2 - 1}{(n/m)^2 + 2}
    \right)
    \nabla I(\vec{r})
\end{equation}

Il risultato netto dei due contributi è che la microsfera tendera ad occupare una
posizione di equilibrio nel punto in cui i due contributi si cancellano e, se
perturbata, risentirà di una forza di richiamo verso la posizione di equilibrio.

Una risultato qualitativamente identico è dimostrabile nel limite dell'ottica
geometrica, quando la particella è al contrario di dimensioni molto maggiori
alla lunghezza d'onda intermedia.

Il caso intermedio richiede l'uso della più complessa teoria Lorenz-Mie e spesso
il ricorso a soluzioni numeriche, ma l'idea qualitativa alla base
dell'intrappolamento resta valida.

Nel caso generale i requisiti per un intrappolamento efficace sono quelli di avere
una forza di gradiente maggiore di quella di scattering e una energia cinetica
delle particelle intrappolate sufficientemente bassa (quindi un fluido sufficientemente viscoso).

Per le nostre applicazioni è sufficiente considerare una forza di richiamo del tipo 

\begin{equation}
    \vec{F} = -k(\vec{x}-\vec{x}_{eq})
\end{equation}

Il valore di $k$ per una certa trappola ottica, come vedremo, può essere
determinato attraverso un'apposita procedura di calibrazione che sfrutta
la diffusione della microsfera all'interno della trappola.





\section{Spettroscopia force-clamp}

\section{\textit{Imaging} di singola molecola}