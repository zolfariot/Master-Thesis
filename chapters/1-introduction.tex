\chapter{Introduction}

% Introduction on the importance of mechanotransduction


 %%%%%%%%% %%%%%%%%% %%%%%%%%% %%%%%%%%% %%%%%%%%% %%%%%%%%% %%%%%%%%%

Combining ultra-fast force spectroscopy with single molecule
fluorescence to characterize \textit{in vitro} the interaction
kinetics of isolated complexes of few biological molecules.

% between 

\section{Optical Tweezers for Force Spectroscopy}

Optical Tweezers (OT) are instruments that exploit the
\emph{radiation force} exerted by highly focused Gaussian beams on
dielectric particles to trap and manipulate microscopic objects with
sub-nanometric precision.

This technology relies on the large intensity gradient of a focused
Gaussian beam close to its waist and on the interaction between the
induced electric dipole of the particle and the beam itself.

An experiment to trap synthetic particles and bacteria with Optical
Tweezers was first realized by Arthur Ashkin in 1986\cite{Ashkin:86}
who was eventually awarded with a Nobel Prize in 2018
\emph{``for the optical tweezers and their application to biological
systems''}.

In the limit of particles much smaller than the wavelength of the
incident Gaussian beam, the force exerted on the particles can be
derived from Maxwell's equations in the Rayleigh regime. The particle
is thus considered as a point induced electric dipole with a
polarization vector equals to $\vec{p} = \alpha \vec{E}$, where
$\alpha$ is the particle polarizability and $\vec{E}$ the electric
field at the particle position.

The force on a point-like infinitesimal dipole can be derived from
the Lorentz force law as:
$$ \vec{F} =
  \left( \vec{p} \cdot \vec{\nabla} \right) \vec{E}
  + \frac{d\vec{p}}{dt} \times \vec{B}
$$
Substitution of the polarization vector yields:
$$ \vec{F} = \alpha
\left[
    \left( \vec{E} \cdot \vec{\nabla} \right) \vec{E}
    + \frac{d\vec{E}}{dt} \times \vec{B}
\right]
$$
Which can be rewritten using Maxwell's equation and vector
properties as: 
\begin{equation}
\label{dipole_force}
\vec{F}
= \alpha 
\left[
    \frac{1}{2}\nabla E^2
    + \frac{d}{dt}\left(\vec{E} \times \vec{B}\right)
\right]
\end{equation}
The last term in \ref{dipole_force} is a derivative of a quantity
oscillating at the optical frequency of the beam (\SI{> 1e14}{\Hz}),
thus completely neglectable at our force sampling rate.
This result shows that the dipole force is proportional to the square
of the electric field, and therefore to the intensity of the incident
beam.

\section{}